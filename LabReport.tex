%%=====================================================================================
%%
%%       Filename:  LabReport.tex
%%
%%    Description:  
%%
%%        Version:  1.0
%%        Created:  Thursday 26 November 2015
%%       Revision:  none
%%
%%         Author:  Fahim Khan
%%   Organization:  eSim,FOSSEE,IIT Bombay
%%      Copyright:  Copyright (c) 2015, eSim
%%
%%          Notes:  
%%                
%%=====================================================================================

\documentclass{article}

%Adding packages generally required for Lab report
\usepackage{amsmath} %Use for math
\usepackage{graphicx} %Use for images
\usepackage{color} %Use for color
\setlength\parindent{0pt} % Removes all indentation from paragraphs
\setlength{\parskip}{1.3ex plus 0.5ex minus 0.3ex} %Vertical space between two para

\begin{document}

\begin{center}
    \LARGE \textcolor{green}{EXPERIMENT NO : 1}
             
\end{center}

    Author   : Prof. Jhon\\
    Date     : January 1,2015\\
    Partners : James Smith,Mary Smith\\


%%%%%%%%%%%%%%%%%%%%%Sections%%%%%%%%%%%%%%%%%%%%%%%%
\section*{\textcolor{red}{Aim of the Experiment :}}

Analysis of small signal BJT Amplifier using eSim.
    

\section*{\textcolor{red}{Theory :}}
In theBipolar Transistor the most common circuit configuration for an NPN transistor is that of the
Common Emitter Amplifier circuit and that a family of curves known commonly as the Output
Characteristic Curves, relate the transistors Collector current (Ic), to the Collector voltage (Vce)

All types of Transistor Amplifiers operate using AC signal inputs which alternate between a positive
value and a negative value so some way of “presetting” the amplifier circuit to operate between
these two maximum or peak values is required. This is achieved using a process known as Biasing.
Biasing is very important in amplifier design as it establishes the correct operating point of the
transistor amplifier ready to receive signals, thereby reducing any distortion to the output signal. \par 


The DC load line can be drawn onto these output characteristics curves to show all the possible
operating points of the transistor from fully “ON” to fully “OFF”, and to which the quiescent
operating point or Q-point of the amplifier can be found. The aim of any small signal amplifier is to
amplify all of the input signal with the minimum amount of distortion possible to the output signal,
in other words, the output signal must be an exact reproduction of the input signal but only bigger
(amplified). \par 


To obtain low distortion when used as an amplifier the operating quiescent point needs to be
correctly selected. This is in fact the DC operating point of the amplifier and its position may be
established at any point along the load line by a suitable biasing arrangement. The best possible
position for this Q-point is as close to the center position of the load line as reasonably possible,
thereby producing a Class A type amplifier operation, ie.Vce = 1/2Vcc. Consider the Common
Emitter Amplifier circuit shown below. \par 



\end{document}
